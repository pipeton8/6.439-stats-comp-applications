\documentclass[11pt, english]{article}

\usepackage{multicol}
\usepackage{hyperref}
\usepackage[eng]{felipito}
\usepackage{stfloats}
\usepackage{mathpazo}

\usepackage[margin=0.7in, left=0.7in, right=0.7in]{geometry}

\graphicspath{{./Graphics/}}

% Colors
\definecolor{urlcolor}{rgb}{0,.145,.698}
\definecolor{linkcolor}{rgb}{.71,0.21,0.01}
\definecolor{citecolor}{rgb}{.12,.54,.11}


% Document title
\title{\bf Problem Set 5 \\ Statistics, Computation and
Applications\\[-1ex]}
\author{Felipe del Canto}
\date{November, 2021}
    
\hypersetup{
	breaklinks=true,  % so long urls are correctly broken across lines
    colorlinks=true,
    urlcolor=urlcolor,
    linkcolor=linkcolor,
    citecolor=citecolor,
}

\begin{document}
    
\maketitle
   
\begin{multicols}{2}

\section*{Problem 5.1: Flows and Correlations}

For part (a),

For part (b),

For part (c),

\section*{Problem 5.2: Predicting Trajectories}

For part (a),

For part (b),


\section*{Problem 5.3: Gaussian Processes}

For part (a), consider the squared exponential/RBF covariance function
	$$\kappa(x_{i}, x_{j}) = \sigma^{2}\exp\left(-\frac{(x_{i} - x_{j})^{2}}{2\ell^{2}}\right).$$
If the signal variance is increased, then all things equal, points will have a higher covariance. In other words, the covariance curve (as a function of distance $|x_{i} - x_{j}|$) is higher, the higher $\sigma^{2}$ is. This is shown in

\begin{figure*}
	\caption{Effect of $\sigma^{2}$ and $\ell$ on the RBF covariance function $\kappa(x_{i}, x_{j})$.}

\end{figure*}


For part (b),

For part (c),




\end{multicols}
\end{document}
